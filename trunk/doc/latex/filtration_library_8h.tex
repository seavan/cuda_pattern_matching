\hypertarget{filtration_library_8h}{
\section{Файл D:/Projects/MatchingSolution/source/filtration\_\-lib/filtrationLibrary.h}
\label{filtration_library_8h}\index{D:/Projects/MatchingSolution/source/filtration\_\-lib/filtrationLibrary.h@{D:/Projects/MatchingSolution/source/filtration\_\-lib/filtrationLibrary.h}}
}
\subsection*{Макросы}
\begin{CompactItemize}
\item 
\#define \hyperlink{filtration_library_8h_f5818b4b91675ed322d636bd417d4e25}{FILTRATIONLIBRARY\_\-API}~\_\-\_\-declspec(dllimport)
\end{CompactItemize}
\subsection*{Функции}
\begin{CompactItemize}
\item 
FILTRATIONLIBRARY\_\-API unsigned char $\ast$ \hyperlink{filtration_library_8h_426460b10f7ee354dc2d14bff87485da}{filtration} (unsigned char $\ast$\_\-input, int \_\-width, int \_\-height, int \_\-method, float \_\-coeff)
\end{CompactItemize}


\subsection{Макросы}
\hypertarget{filtration_library_8h_f5818b4b91675ed322d636bd417d4e25}{
\index{filtrationLibrary.h@{filtrationLibrary.h}!FILTRATIONLIBRARY\_\-API@{FILTRATIONLIBRARY\_\-API}}
\index{FILTRATIONLIBRARY\_\-API@{FILTRATIONLIBRARY\_\-API}!filtrationLibrary.h@{filtrationLibrary.h}}
\subsubsection[{FILTRATIONLIBRARY\_\-API}]{\setlength{\rightskip}{0pt plus 5cm}\#define FILTRATIONLIBRARY\_\-API~\_\-\_\-declspec(dllimport)}}
\label{filtration_library_8h_f5818b4b91675ed322d636bd417d4e25}




\subsection{Функции}
\hypertarget{filtration_library_8h_426460b10f7ee354dc2d14bff87485da}{
\index{filtrationLibrary.h@{filtrationLibrary.h}!filtration@{filtration}}
\index{filtration@{filtration}!filtrationLibrary.h@{filtrationLibrary.h}}
\subsubsection[{filtration}]{\setlength{\rightskip}{0pt plus 5cm}FILTRATIONLIBRARY\_\-API unsigned char$\ast$ filtration (unsigned char $\ast$ {\em \_\-input}, \/  int {\em \_\-width}, \/  int {\em \_\-height}, \/  int {\em \_\-method}, \/  float {\em \_\-coeff})}}
\label{filtration_library_8h_426460b10f7ee354dc2d14bff87485da}


Фильтрация изображения заданным методом (CPU) \begin{Desc}
\item[Аргументы:]
\begin{description}
\item[{\em \_\-input}]Построчный массив с исходным изображением \item[{\em \_\-width}]Ширина исходного изображения \item[{\em \_\-height}]Высота исходного изображения \item[{\em \_\-method}]Метод (0 - Собель, 1 - Пюритт, 2 - Лаплас) \item[{\em \_\-scale}]Резкость фильтрации \end{description}
\end{Desc}
\begin{Desc}
\item[Возвращает:]Массив байт с отфильтрованным изображением с глубиной цвета - 8 (grayscale) \end{Desc}
